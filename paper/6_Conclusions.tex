\section{Conclusions}\label{s:conclusion}

\todo{KIE: Mention our ResiCAV experiments and the importance to recognize that once physics is involved then a single experiment is not sufficient to determine the outcome / impact of a test. Instead, at this stage several thousands of tests may need to be executed to obtain reliable statistics to estimate the worst/best case.}

We have provided a summary of the sources of non-determinisim in game engines. This was followed by presenting a method by which one can determine the simulation variance of a gaming engine used in V\&V application of CAV simulations. The method consists of five stages, starting with designing the experiment, then suggesting the various settings that can improve or worsen the performance, and finally focuses on running the experiment and analysing the output.

A case study is then used to show the implementation of the method introduced, where several tests were setup and run to explore the different interactions between the different agents in a simulator called CARLA. It was found that for the computer used (with specs of an i9 9960X CPU chip, NVIDIA GeForce RTX 2080 graphics card and 64GB of RAM) the simulation variance was sufficiently low to assume the simulator to be deterministic as long as the CPU + GPU utilisation is below 50\% and that tests are terminated as soon as a collision of any type is detected. %since it was noticed that collisions cause the level of non determinism to jump remarkable invalidating the assumption of the simulator is deterministic.

In highlight of obtaining a fully determinstic simulator, one would either have to explore methods to control the scheduling of threads in a computer such that if tests are rerun the utilisation of hardware will be the same. Alternatively, another approach would be to create a new engine that is designed to be deterministic with minimal attention given to performance. 

