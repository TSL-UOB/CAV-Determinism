\title{Investigating Determinism of Physics Engines for usage in CAV and V\&V simulations}


\author[1]{Abanoub Ghobrial\thanks{$^{1}${\footnotesize \{abanoub.ghobrial, greg.chance, kerstin.eder\}@bristol.ac.uk}}}
\author[1]{Greg Chance}
\author[2]{Severin Lemaignan\thanks{$^{2}${\footnotesize \{severin.lemaignan, tony.pipe\}@brl.ac.uk}}}
\author[2]{Tony Pipe}
\author[1]{Kerstin Eder}

\affil[1]{Trustworthy Systems Lab (TSL), Department of Computer Science, University of Bristol}
\affil[2]{Bristol Robotics Lab (BRL), University of West of England}




\maketitle

\begin{abstract}
\noindent The industry and certification bodies for connected autonomous vehicles is adopting the use of physics and gaming engines to develop, train, verify, validate and certify the software of these autonomous systems in simulation. It is important for these engines to be deterministic in order for one to carry out these different processes. However, such engines are inherently non-deterministic. We propose a method by which one can identify the region where they can operate in these engines and guarantee that the level of non determinism is sufficiently low that it can be assumed to be deterministic. We also implement this method on a case study and show how we identified the region (encompassing computational utilisation and behaviours in simulation) at which one can guarantee performance to be sufficiently deterministic.          
\end{abstract}

\begin{IEEEkeywords}
Autonomous Driving, Determinism, Physics Engines, Connected Autonomous Vehicles (CAV), Verification and Validation(V\&V)
\end{IEEEkeywords}
\IEEEpeerreviewmaketitle