

\section{Introduction} \label{s:introduction}

% What is verification
\IEEEPARstart{V}{}erification is the process used to gain confidence in the correctness of a system with respect to its requirements~\cite{bergeron2012writing}. Simulation based verification can be used to test autonomous driving functions under development benefiting from full control over the road network and the actors within it. These simulated tests aim to provide evidence to regulators of the functional safety of the vehicle or its compliance with commonly agreed upon road conduct~\cite{ViennaConv}, national rules~\cite{codes2015highway} and road traffic laws~\cite{RoadTraffic1988} which form a body of driving assertions which must not be violated. Design confidence is gained when the autonomous vehicle (AV) can be shown to comply with these rules through assertion checking. 
%
There have been a number of fatalities with autonomous vehicles which could be attributed to a lack of verification and validation (V\&V), e.g.\cite{FatalityExample}.
%
Simulation is an obvious domain choice to explore the vast parameter space in a safe and efficient manner~\cite{korosec2019waymo} without millions of miles of costly on-road testing~\cite{kalra2016driving} where the rare event can be made to happen more often~\cite{Koopman2018} such as unpredictable actor behaviour~\cite{RobustnessAutonomy}. 

% Requirements of simulation based verification SBV
For games engines to offer a suitable testing environment they must operate with a minimal \textit{reality gap} in the physical domain (vehicles, actors) and be realistic enough \cite{Koopman2018} to produce suitable rendering of scenes for perception stack testing. 

The simulation must be easy to setup, run and manipulate the environment and the temporal development of actors \cite{Ulbrich2015} and include a provision for sensor analogues to a suitable degree of accuracy (e.g. video, LIDAR, IR). Some researchers may also require support for hardware-in-the-loop development or a suitable real-time test-bed for cyber-security testing~\cite{Javaid2013}. 
%, e.g. minimum rate of frames per second (fps).
%
% Game engines for SBV
Game engines offer a simulation domain solution for the development and testing of CAVs that meet many of these requirements, as opposed to vehicle dynamics simulators and traffic-level simulators used by manufacturers~\cite{FrameworkAndChallenges}; but many challenges still exist.

% Verification and Determinism 
For game engines to be useful as a verification tool then they must be deterministic. 
%
% What is a deterministic system
Deterministic, in this context, refers to the property of causality given a temporal development of events such that any state is completely determined by prior states.
%
Herein the term \textit{simulation variance} is used to refer to how simulation outputs vary when repeated many times and specifically the variance of actor paths within a given scenario. We will adopt the terminology defined in~\cite{Ulbrich2015}, where \textit{scene} refers to all static objects including the road network, street furniture, environment conditions and a snapshot of any dynamic elements. Dynamic elements are the elements in a scene whose actions or behaviour may change over time, these are considered actors and may include other road vehicles, cyclists, pedestrians and traffic signals. The \textit{scenario} is then defined as a temporal development between several scenes which may be specified by specific goals and values. A \textit{situation} is defined as the subjective conditions and determinants for behaviour at a particular point in time.

% Why is determinism important for verification
Determinism is important for verification. Any software defects that are detected but not repeatable at a later time or by another party does not allow for suitable resolution. Furthermore, defects not found on the verification testing system may arise on a deployed system with a different hardware architecture resulting in untested and potentially unsafe behaviour. %When considering the development of autonomous vehicles, untested code could lead to dangerous behaviours not being identified and fatal road accidents may occur.
%
Also, if the simulation is non-deterministic, variance in actor positions may lead to subsequent assessment errors, e.g. assertion checking.

% Challenges of SBV
Our main research question is to what extent are gaming engines deterministic so that they can provide a reliable and repeatable testing environment for CAV verification? 
%
The game engine, in this context, must also comply with the requirements to provide a suitably realistic environment and real-time processing for perception and security analysis.

This paper is structured as follows. In Section~\ref{s:background} a background on gaming engines and sources of potential non-determinism are explored. Section~\ref{s:methodology} presents a method to design and execute an experiment to assess suitability of a system for CAV verification. A case study to illustrate how the simulation variance vary for a number of scenarios is explored in Section~\ref{s:case-study}. We conclude in Section~\ref{s:conclusion} and give an outlook on future work.


% \noindent \IEEEPARstart{T}{}he evolution and inevitable implementation of connected and autonomous vehicles (CAVs) on roads is a delicate if not utterly fascinating subject. Such systems operate in environments marked by inaccessibility, unexpected weather conditions, or unpredictable behavior by surrounding humans\cite{RobustnessAutonomy}. 

% Public safety is clearly a prime concern for putting autonomous vehicles into play, which means the meticulous and thorough process of developing, training, verifying, validating and certifying the artificial intelligence (AI) and software of the vehicle is a task to be handled with care. 

% It requires lots of flexibility and room to iterate to ensure the algorithms are given the sharpest understanding for maximum performance as well as to ensure that they are hazard-free in order for them to be deployed\cite{AirsimUnrealArticle}.\\\\

% \noindent There has been a number of catastrophic fatalities with self-driving vehicles due to lack of verification and validation (V\&V) of software e.g.\cite{FatalityExample}.

% As such, there appears the need for finding ways to V\&V the safety of such software, ranging from AI development to cyber-security, while maintaining minimal risks in testing.\\\\

% ======================
% This is all from "FrameworkAndChallenges" without using quotes 
% \noindent Existing vehicle manufacturers use vehicle dynamic simulators to model the physics of each component in a system. Similarly, the use of simulation for traffic modelling is well established. Despite these proven simulation tools, non of them provide all of the required functionalities together to fully simulate CAVs\cite{FrameworkAndChallenges}.\\\\

% While traffic-level simulators can model road layouts, they do not require as much realism or detail as CAVs require. On the other side, vehicle dynamics simulators include some of the high fidelity physics functionalities needed for CAV testing, but they only consider the vehicle itself and perhaps the road surface and gradient; they do not model any of the road layout, signs, traffic/street lights, other street furniture or surrounding infrastructure. 
% Most critically, neither types of simulators model sensors such as cameras, lidars or radars\cite{FrameworkAndChallenges}.\\\\ 
% Consequently, one of the major routes CAVs developers are moving towards is using physics engines, these are often games engine development frameworks, since they tick all of the boxes needed to fully simulate CAVs \cite{FrameworkAndChallenges}.  
% ======================

% Robopilot, Capri, Carla, Apollo, Airsim, Udacity etc. are all examples of driving simulator projects that use physics engines to support development, training, and validation of autonomous driving systems software\cite{ListOfSimulators}. \\\\
% Test-retest reliability, is a key factor in V\&V of software. Tests need to be repeatable in order for one to eliminate bugs. 
% In other words physics engines has to be deterministic, or the level of non-determinism of the engine should be low and clearly defined, for them to be used in the context of V\&V of CAV software.\\\\
% \TODO{Remove this paragraph?} Determinism of gaming (physics) engines has not attracted much attention by engines' developers nor researchers. This previously was not important since in gaming, determinism is not critical. 
% Performance, on the other hand, is a subject of more interest in that field, where developers want games to work on all platforms and still be fast to provide a positive user experience. 
% On the contrary, determinism becomes vital and performance not when such engines are used in autonomous vehicles' AI development and testing. \\\\

% This paper aims to investigate how deterministic physics engines are for usage in V\&V for CAV simulations; and will cover the list of following points:
% \begin{itemize}[leftmargin=*]
%     \item Necessary background on gaming engines used in CAV simulations.
%     \item Review of sources of non-determinism in such engines.
%     \item Provide a method of how the level of non-determinism can be investigated and defined.
%     \item A case study showing the implementation of this methodology in an engine widely used in CAVs simulations (Unreal Engine).
% \end{itemize}
 
 



