\section{Methodology} \label{s:methodology}
\noindent A method for determining the level of simulation variance of an engine is presented in this section and the working flow of the method is shown in Figure~\ref{method_diagram}. Each of the steps in the method is elaborated in the following subsections.
%
\subsection{Design experiment}
The experiment should be a simple scenario that can be easily repeated with the same initial conditions. Logs should be used to record the actor positions at fixed time intervals throughout the simulation. For each actor the variance in trajectory can be determined and the deviation from the mean, $S_t$, of the trajectory monitored at each time step. Averages over time and for different actors can be used to condense results down to single figures.
Tests should be run to a significant number of times (e.g.100-1000), to make sure any edge variations are encountered. 

We use game engine actors to simulate the AVs and other agents (vehicles and pedestrians). Actors can either follow a series of trajectory waypoints or can plan a route to a further destination using some path planning and obstacle avoidance algorithms, e.g. A* search algorithm. 
% 
% Resource - Realism
To simulate the demands of realistic rendering and complex physics calculations in a simple scenario, stress routines are used to artificially occupy system resources. 
% Collision callbacks
Further to this, the experiment should be designed to create collision callbacks to the physics engine by allowing actors to collide during the scenario. 
%
The user should also design the experiment to monitor and record system CPU and GPU utilisation. Additionally, if the user is interested in real-time assessment of the simulation then FPS should also be monitored and logged.
%
Where possible a fixed $dt$ should be used and any random numbers used should be seed controlled.
% \noindent When designing the experiment(s) one needs to bare in mind the following two questions: i) Is the engine deterministic? ii) If not, how does its simulation variance vary? In simulation, actors can either follow a series of trajectory way points or they get given the destination and their AI path plans to that destination. Thus, given that an engine has a high level of simulation variance, then one needs to set up an experiment that would stress the engine in order to determine how its simulation variance vary. 
% This is best done by generating collision callbacks, because that is when engines' do a lot of calculations to determine the response after the collision. Another approach, if AI is used for path planning instead of just following predefined trajectory way points, is to introduce a decision point for the AI.\\\\
It is crucial to also make sure that any randomisation factors are eliminated in the experiment.
%
\subsection{Internal settings}
There are several error and physics collision internal settings in physics engines that can be tweaked to enhance the simulation variance of the engine.

Increasing the physics time-step calculations, which is increasing the number of calculations per unit time, would improve the level of simulation variance since the physics sequence will be more finely defined; but on the other hand this would be more computationally expensive. However, in such applications of V\&V of CAV simulations, the computational cost should be of less concern compared to the importance of repeatability of tests.

If the experiment setup includes the usage of the physics engine's AI then altering the navigation mesh settings in the engine, like increasing the granularity of the mesh or changing the mesh type; can improve the level of simulation variance as a result of allowing the AI to navigate in a more well defined space. 

Disabling rendering or running in headless mode is a way of attempting to entirely decouple the rendering engine from the physics engine, which theoretically should affect positively the repeatablility of tests. 
The downside of most physics engines is that they do not provide a headless mode in the editor setup, thus this could make running in headless mode a real challenge for verification engineers.

Running in editor or deployed mode can sometimes cause massive differences in performance, with deployed mode being worse, mostly due to the way settings get packaged when in deployed mode.

\begin{table*}[b]
\centering
\begin{tabular}{clclcc}
\toprule
Test ID & Test description (See Figures~\ref{Test_a} and \ref{Test_b}) & Collision/Intersection & Collision Type & Look ahead distance (m) & No. of repeats \\ \midrule
1       & Two vehicles driving                   & No  & N/A & 2 & 1000 \\
2       & Two vehicles driving                   & Yes & Vehicle and Vehicle & 2 & 1000 \\
3       & Two vehicles driving and a pedestrian  & No  & N/A & 2 & 1000 \\
4       & Two vehicles driving and a pedestrian  & Yes & Vehicle and Pedestrian & 2 & 1000 \\
5       & Two pedestrians                        & No  & N/A & 2 & 1000 \\
6       & Two pedestrians                        & Yes & Pedestrian and Pedestrian & 0.4 & 1000 \\
7       & Two pedestrians                        & Yes & Pedestrian and Pedestrian & 2 & 1000 \\
8       & Two pedestrians                        & Yes & Pedestrian and Pedestrian & 20 & 1000 \\
\bottomrule
\end{tabular}
\caption{Set of experiments}
\label{TableOfExperiments}
\end{table*}

\subsection{External settings}
Running other programs in the background, like having a web browser open, or running the physics engine in the background while performing other tasks on the same machine does also have an effect; because it alters the CPU and GPU utilisation of tasks.

Trying to run stress experiments on computers by running other programs makes it difficult to control experiments since the utilisation would be inconsistent, nonetheless, there exists dedicated stress utilisation programs which provide consistent CPU and GPU stressing. Note that during running these tests computers should be left alone and not to be used for any other purposes apart from running the experiments, else the CPU + GPU stressing will lose its consistency through out a given test. 

\subsection{Running experiment}
\noindent When it comes to running the experiment itself, things that one should consider are the problem space exploration; the number of runs that should be executed to get reliable results; the frequency of logging data of actors; and possibly having a program to monitor the hardware utilisation.
% Problem exploration, there is too many variables that one can play with. do statistical analysis some.... some of them get problematic...

\subsection{Output and analysis}
Once the experiments are run, the logged data is post processed to find the variance in the logged data between the different runs. Note that bifurcation effects can cause a jump in the variance, so it is worth plotting the different logged paths in order to determine if there are any bifurcation effects.

This whole process is repeated for various internal and external settings (if needed). 
Then various plots can be created to draw the engine's simulation variance. 
Thereon, it would be up to the verification engineers to determine where they want to define the line below which the simulation variance would be acceptable, and thus know at which levels of computational utilisation they can guarantee to run repeatable experiments.   

