%%%%%%%%%%%%%%%%%%%%%%%%%%%%%%%%%%%%%%%%%
% Professional Formal Letter
% LaTeX Template
% Version 2.0 (12/2/17)
%
% This template originates from:
% http://www.LaTeXTemplates.com
%
% Authors:
% Brian Moses
% Vel (vel@LaTeXTemplates.com)
%
% License:
% CC BY-NC-SA 3.0 (http://creativecommons.org/licenses/by-nc-sa/3.0/)
%
%%%%%%%%%%%%%%%%%%%%%%%%%%%%%%%%%%%%%%%%%

%----------------------------------------------------------------------------------------
%	PACKAGES AND OTHER DOCUMENT CONFIGURATIONS
%----------------------------------------------------------------------------------------

\documentclass[11pt, a4paper]{letter} % Set the font size (10pt, 11pt and 12pt) and paper size (letterpaper, a4paper, etc)

\usepackage{graphicx} % Required for including pictures

\usepackage[T1]{fontenc} % Output font encoding for international characters
\usepackage[utf8]{inputenc} % Required for inputting international characters
\usepackage{gfsdidot} % Use the GFS Didot font: http://www.tug.dk/FontCatalogue/gfsdidot/
\usepackage{microtype} % Improves typography


% \input{structure.tex} % Include the file that specifies the document structure
\pagestyle{empty} % Suppress headers and footers

% \setlength\parindent{1cm} % Paragraph indentation

% Create a new command for the horizontal rule in the document which allows thickness specification
\makeatletter
\newcommand{\vhrulefill}[1]{\leavevmode\leaders\hrule\@height#1\hfill \kern\z@}
\makeatother

%----------------------------------------------------------------------------------------
%	DOCUMENT MARGINS
%----------------------------------------------------------------------------------------

\usepackage{geometry} % Required for adjusting page dimensions

\geometry{
	top=1cm, % Top margin
	bottom=1.5cm, % Bottom margin
	left=3cm, % Left margin
	right=3cm, % Right margin
	%showframe, % Uncomment to show how the type block is set on the page
}

%----------------------------------------------------------------------------------------
%	DEFINE CUSTOM COMMANDS
%----------------------------------------------------------------------------------------

\newcommand{\logo}[1]{\renewcommand{\logo}{#1}}

\newcommand{\Who}[1]{\renewcommand{\Who}{#1}}
\newcommand{\Title}[1]{\renewcommand{\Title}{#1}}

\newcommand{\headerlineone}[1]{\renewcommand{\headerlineone}{#1}}
\newcommand{\headerlinetwo}[1]{\renewcommand{\headerlinetwo}{#1}}

\newcommand{\authordetails}[1]{\renewcommand{\authordetails}{#1}}

%----------------------------------------------------------------------------------------
%	AUTHOR DETAILS STRUCTURE
%----------------------------------------------------------------------------------------

\newcommand{\authordetailsblock}{
	\hspace{\fill} % Move the author details to the far right
	\parbox[t]{0.48\textwidth}{ % Box holding the author details; width value specifies where it starts and ends, increase to move details left
		\footnotesize % Use a smaller font size for the details
		\Who\\ % Author name
		\textit{\authordetails} % The author details text, all italicised
	}
}

%----------------------------------------------------------------------------------------
%	HEADER STRUCTURE
%----------------------------------------------------------------------------------------

\address{
	\includegraphics[width=2.1in]{\logo} % Include the logo of author institution
	\hspace{0.62\textwidth} % Position of the institution logo, increase to move left, decrease to move right
	\vskip -0.1\textheight~\\ % Position of the large header text in relation to the institution logo, increase to move down, decrease to move up
	% \Large\hspace{0.2\textwidth}\headerlineone\hfill ~\\[0.006\textheight] % First line of institution name, adjust hspace if your logo is wide
	\Large\hspace{0.4\textwidth}\headerlineone\hfill ~\\[0.006\textheight] % First line of institution name, adjust hspace if your logo is wide
	\hspace{0.4\textwidth}\headerlinetwo\hfill \normalsize % Second line of institution name, adjust hspace if your logo is wide
	\makebox[0ex][r]{\textbf{\Who\Title}}\hspace{0.01\textwidth} % Print author name and title with a little whitespace to the right
	~\\[-0.01\textheight] % Reduce the whitespace above the horizontal rule
	\hspace{0.4\textwidth}\vhrulefill{1pt} \\ % Horizontal rule, adjust hspace if your logo is wide and \vhrulefill for the thickness of the rule
	\authordetailsblock % Include the letter author's details on the right side of the page under the horizontal rule
	\hspace{-0.25\textwidth} % Horizontal position of the author details block, increase to move left, decrease to move right
	\vspace{-0.1\textheight} % Move the date and letter content up for a more compact look
}

%----------------------------------------------------------------------------------------
%	COMPOSE THE ENTIRE HEADER
%----------------------------------------------------------------------------------------

\renewcommand{\opening}[1]{
	{\centering\fromaddress\vspace{0.05\textheight} \\ % Print the header and from address here, add whitespace to move date down
	\hspace*{\longindentation}\today\hspace*{\fill}\par} % Print today's date, remove \today to not display it
	{\raggedright \toname \\ \toaddress \par} % Print the to name and address
	\vspace{1cm} % White space after the to address
	\noindent #1 % Print the opening line
}
%----------------------------------------------------------------------------------------
%	SIGNATURE STRUCTURE
%----------------------------------------------------------------------------------------

\signature{\Who\Title} % The signature is a combination of the author's name and title

\renewcommand{\closing}[1]{
	\vspace{2.5mm} % Some whitespace after the letter content and before the signature
	\noindent % Stop paragraph indentation
	\hspace*{\longindentation} % Move the signature right to the value of \longindentation
	% \hspace*{-6.5cm}\includegraphics[width=1in]{signature.png}
	\parbox{\indentedwidth}{
		\raggedright
		#1 % Print the signature text
		\vskip 1.65cm % Whitespace between the closing text and author's name for a physical signature
		\fromsig % Prints the value of \signature{}, i.e. author name and title
	}
}
\longindentation=0pt % Un-commenting this line will push the closing "Sincerely," and date to the left of the page

%----------------------------------------------------------------------------------------
%	YOUR INFORMATION
%----------------------------------------------------------------------------------------

\Who{Dr. Greg Chance} % Your name

\Title{, PhD, CEng, MInstP} % Your title, leave blank for no title

\authordetails{
	Trustworthy Systems Lab\\ % Your department/institution
	University of Bristol\\
	1 Cathedral Square\\ % Your address
	Bristol, BS1 5DD, UK\\ % Your city, zip code, country, etc
	Email: greg.chance@bristol.ac.uk\\ % Your email address
	URL: \url{bristol.ac.uk}\\ % Your URL
	% URL: \url{bristol.ac.uk/engineering/research/trustworthy-systems-laboratory/}\\ % Your URL
	% YouTube: \url{https://youtu.be/1emO3-OYn5k}
}

%----------------------------------------------------------------------------------------
%	HEADER CONTENTS
%----------------------------------------------------------------------------------------

\logo{logo.png} % Logo filename, your logo should have square dimensions (i.e. roughly the same width and height), if it does not, you will need to adjust spacing within the HEADER STRUCTURE block in structure.tex (read the comments carefully!)


\headerlineone{} % Top header line, leave blank if you only want the bottom line
\headerlinetwo{} % Bottom header line



%----------------------------------------------------------------------------------------

\begin{document}

%----------------------------------------------------------------------------------------
%	TO ADDRESS
%----------------------------------------------------------------------------------------

\begin{letter}{
	Prof. Eskandarian\\
	Department Head\\ 
	Nicholas and Rebecca Des Champs Chair\\
	Mechanical Engineering Department, Virginia Tech\\
	635 Prices Fork Road, 449 Goodwin Hall (MC 0238)
	Blacksburg, VA 24061
}

%----------------------------------------------------------------------------------------
%	LETTER CONTENT
%----------------------------------------------------------------------------------------

% \opening{}


\textbf{Review of paper T-ITS-21-05-1111}.

Dear Prof Azim Eskandarian,

Thank you for taking the time to review our manuscript. We have given each comment careful consideration between all the authors and we have made changes to the document that make the document clearer and more correct. We give a full account of the changes and rebuttals below and we include a pdf diff to show the changes made.

\textbf{Reviewer 1}\\
\textit{Comment 1: "the whole manuscript seems written from the perspective of Game Engines testing, which lacks practical values in the transportation area."}.

Thank you for this comment, we believe this comment requires a considered response to ensure the main message of the paper comes across to the reader. Carla is based on a game engine and Carla is currently a popular choice for simulation-based vehicle testing, so this would suggest that the underlying  game engine of Carla is of practical value to the community of intelligent transportation. The paper is not focused on game engines per se, but rather the implications of using those game engines for vehicle testing. 

Furthermove, we state on pg.4 the differences between the requirements of game engines for gaming, and for that of AV testing. "Considering the objectives for gaming and comparing them to these for AV development and testing, there are fundamental differences. Providing game players with a responsive real-time experience is often achieved at the cost of simulation accuracy and precision. The gamer neither needs a faithful representation of reality (i.e. gamer accepts low accuracy) nor require repeated actions to result in the same outcome (i.e. gamer accepts low precision). In contrast, high accuracy and precision are necessary for AV development, testing and verification." 

Carla, and other game engine based simulators, will be an entry point for many SME's and start-up companies looking to develop products and services in this area and we believe that this paper brings pertinent information to this community, many of whom may look to ITS for guidance.

Given the confusion to reviewer 1 of our message, we believe that the title of the document should be changed to better reflect the content of the paper, which is primarily concerned with the importance of determinism for simulation based verification of autonomous vehicles. Hence we have changed the paper title to: "On the Importance of Determinism for Simulation-based Autonomous Vehicle Verification using a Game Engine".

\textit{Comment 2: "One major concern is index selection. As we all know, the scale of deviation relies on the mean values of the investigated variables. Thus, I wonder why the authors pick maximum deviation to measure the performance of the simulation results. In my opinion, the average deviation seems better in measuring the overall performance for the whole simulation process."}.

The issue we found with taking the average deviation over many (100's or 1000's) of repeated runs is that a single 'failure' can be hidden in an average. For example 1 error in 1000 would be an insignificant difference to an average, but a single simulation run that exceeds permisible variance could result in a false negative result or even fail to detect the presence of a serious fault with the system under test. The verification process needs to be aware of any failure, as even a single failure may coincide with a bug or error in the system that needs to be found and corrected. Hence, it is imperative that we discover any simulation that exceeds the simulation variance. We could reframe this as thinking less about simulation performance, and more about detecting any errors in the system. Verification requires the detection of edge cases (rare cases) not the average performance of the system.

We have added (in italics) to the sentence on page 9:  "Initial testing [26] indicated an actor path deviation of1x10-13cm for 997 out of 1000 tests, with three tests reporting a deviation of over 10cm. \textit{Due to the low probability of fault rates observed, using an average simulation variance would 'hide' these errors and hence this is the reason that we use maximum variance and not an average}". 

\textit{Comment 3: "Moreover, I also concern about the results from several tests. As stated in Section V-B, 1 cm considered sufficient for urban scenario assertion checking. However, the average vehicle speeds are setting as 20–35 km/h (i.e. 5.56-9.72 m/s) [1]. Thus, I doubt the practical value of this permissible variance (too accurate for general AV testing) in real-world autonomous vehicle verification. The authors should provide more support materials for the setting used in the whole experiment"}.

The focus of this paper is not really the absolute values that we chose for permissible variance or the speed of the vehicles (which was based on UK urban speed limit of 20mph or 30kmph), but more about the method of detecting and controlling simulation variance in an end-user's system. The value of permissible variance chosen in the paper (1cm) was chosen as a good point between, say 1m (too coarse) and 1mm (too fine), to illustrate the violation of this limit given certain simulator and system conditions.

The point is that repeated runs should give the exact same output, regardless of permissible variance or simulated vehicle speed. The reviewer suggests a paper that descibes the speed of vehicles in an urban environment similar to those in our case study. We have included this reference including a line to point interested readers whom wish to set the speed of their simulated vehicles to an apporpriate value.

Please see the additional line on page 7: \textit{This tolerance may need to be chosen for the specific verification case and the speed of the vehicles within the environment [1]}.

\textit{Comment 4: "The whole paper seems much more relevant to the UE4 engine than the CARLA platform. I understand the CARLA application is just a case study, but the first half part of the manuscript is less relevant to the AV testing. In particular, the authors said “the AV simulation domain introduces its unique challenges that were not considered in that paper.” in section IV"}.

Carla uses Unreal Engine 4 (UE4) for vehicle testing, which is not the original purpose of UE4. The issues discussed in the paper do not have any negative implications for games, but (as the paper argues) they do have negative implications for vehicle testing. Thus, the issues are issues with Carla rather than UE4, and the same issues are likely to arise in any simulator based on a game engine.

\textit{Comment 5: "The numbers of references are odd"}.
Please let us know any format changes that are required specifically. If you are referring to the document hyperlinks that link to each reference, these can be switched off if required.

\textit{Comment 6: "In section I, the sentence “without the need for millions of miles of costly on-road testing” is too arbitrary, the on-road testing is still necessary for AVs verification"}.
We agree with the reviewer that the need for on-road testing is an essential component of the verification process. However, the previous sentence highlights the shortcomings of on-road testing where human fatalities have occured due to a lack of exploration of the parameter space.
The point of this sentence  was in the context of parameter space exploration and the fact that this space could never be explored entirely with on-road testing alone and hence the neccessity of simulation for verification.

\textit{Comment 7: "No need to provide such a detailed description of floating-point arithmetic since it does not cause non-zero simulation variance"}.
Floating point arithmetic does not contribute to simulation variance, but it is a common misconception that people often jumpt to when considering or discussing deterministic computation. We therefore believe it is important to fully explain the issue to ensure that there is no doubt going forward from this point.

To make this clearer we have modified this section to indicate this misconception. We have also made this section more concise. Please see page 4-5 starting \textit{A common misconception that is often jumped to when concerning non-determinism is the use of floating point number representation. This is an erroneous conflation with non-deterministic computational execution and we explain the reasons for this here}.

\textit{Comment 8: "There is no equal sign in Equation (1)"}.
Thank you we have created a symbol for this to turn this into an equation. See the amended equation on page 8.

\textit{Comment 8: "TABLE I and Fig 3 are not consistent, it would be better if there are three subfigures in Fig 3"}.
Thank you we agree this would be clearer if there were three figures, this was done to save space in the document. Please see page XX for the new figure.










Thank you and the reviewers for your time and consideration into this paper. 

\closing{Sincerely,}

% \begin{figure}
% 	\includegraphics[width=1in]{signature.png}
% \end{figure}


%----------------------------------------------------------------------------------------
%	OPTIONAL FOOTNOTE
%----------------------------------------------------------------------------------------

% Uncomment the 4 lines below to print a footnote with custom text
% \def\thefootnote{}
% \def\footnoterule{\hrule}
% \footnotetext{\hspace*{\fill}{\footnotesize\em Footnote text}}
% \def\thefootnote{\arabic{footnote}}

%----------------------------------------------------------------------------------------

\end{letter}

\end{document}
